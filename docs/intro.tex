\section{Introduzione}

Lo sviluppo di un'applicazione tipicamente prevede i seguenti passi:
\begin{itemize}
\item Analisi del problema
\item Progettazione
\item Implementazione
\end{itemize}

L'analisi del problema fornisce come risultato delle specifiche precise e non ambigue sulle quali basare la progettazione.
Seguendo il paradigma della programmazione orientata agli oggetti, durante la progettazione 
vengono definite le \emph{classi} e le relazioni che intercorrono tra di loro.
Tale sistema viene rappresentato mediante un diagramma \emph{UML} e implementato in un linguaggio orientato agli oggetti. 

Partendo dal problema di realizzare un sistema di tutoraggio per studenti universitari, abbiamo affrontato i passi sopra descritti fino arrivare a un'effettiva implementazione.

\subsection{Descrizione del problema}
\label{sec:DescrizioneDelProblema}

Un \emph{servizio di tutoraggio per studenti universitari} � un sistema tramite il quale gli \emph{studenti} possono usufruire dell'assistenza di un gruppo di \emph{tutor}.

Ogni studente interessato pu� iscriversi facendone richiesta a un \mbox{\emph{amministratore}}.

I servizi che sono forniti agli studenti possono essere diversi, come:
\begin{itemize}
	\item un \emph{forum} formato da pi� \emph{gruppi di discussione}, uno per ogni materia.
	\item una bacheca
	\item posta elettronica
	\item una collezione di domande frequentemente poste
	\item e altro...
\end{itemize}

Nel nostro progetto abbiamo sviluppato il forum, in quanto parte fondamentale del sistema.
